
CHAPTER. 1. NUMBERS SYSTEMS 5
Adding the 22 column, we obtain (1)2 and so this time there is nothing to carry. Adding the
23 column, we obtain (11)-3 and so we record 1 and carry 1 into the 24 column. Adding the
2" column, we obtain (1)9 and the process terminates,
Thus we obtain (1011)-3-:-(1111);-1-(11); = (11101); In decimal numbers, this addition is 11+15+3,
and the sum we obtained is 29. El
Binary subtraction
Let us first revise how We deal with “borrowing” in base 10.
Example 1.6 \¢Ve perform the subtraction 4003 —~ 597.
oa 4:-0:
rzscnocz
ooozo
c>=1c»:€_I,
Since 7 is greater than 3, we must borrow a 10. In this case, we do this from the 103 column. We
rewrite the borrowed 1000 as 990 + 10. So we reduce the entry in the 103 column from 4 to 3,
replace the zeros by 9’s in the next two columns and replace 3 by 10‘+3 in the units column. The
resulting row is recorded above the horizontal line. We can now obtain the answer by subtracting
597 from the row above the horizontal line. U
We shall now carry out the same process in binary, where we have to borrow a. 2 when we have to
perform the subtraction (0)2 ~ (1)2 in any column. The (0)3 then becomes (10);, and we have
(10)2 ~ (1)2 = (1)2-
This process is illustrated in the example below.
Exaniple 1.7 ll/e subtract (110l000)9 — (l01011)2.
Step 1
1 1 0 0 1 l 10
1 1 0 1 0 O O
1 0 1 0 1 1
As in the previous example, we have to borrow to perform the subtraction in the units
column. The first column with a non-zero entry is the 23 column. We borrow (1000); from
this column and rewrite it as (110); + (10); (effectively, we are borrowing an S and rewriting
it as 6 + 2). The revised row is recorded above the horizontal line.
Step 2 We replace the row representing the number we are subtracting from by the row above
the horizontal line at the end of Step 1. We can now perform the subtraction in the first
three columns as shown below. When we reach the 23 column; we have to borrow again. We
repeat the procedure described in Step 1. The revised top row is shown above the horizontal
line.
1 0 1 10
1 1 0 0 1 1 10
101011
101
Step 3 We replace the row representing the number we are subtracting from by the row above
the horizontal line and perform the subtraction in the 23 and 24 columns. When we reach
the 25 column, we have to borrow from the next column. The revised top row is shown above
the horizontal line‘
0 10
1 0 1 10 1 1 ,10
1 0 1 0 1 1
11101




CHAPTER 1. NUMBERS SYSTEMS 6
Step 4 \Ve replace the row representing the number we are subtracting from by the row above
the horizontal line and complete the subtraction.
l0l101ll0
101011
111101
Thus we have obtained (1101000)? - (10lO1l)2 2 (111101)-5. In decimal numbers, this subtraction
is 104 -- 43 and the answer we obtained is 61. El
Binary multiplication
The process of binary multiplication is very similar to long multiplication in the decimal system.
A familiar feature in decimal is that \vhen we multiply any whole number by 10, we move all the
digits one place to the left and enter a zero in the units column. Let us pause a moment to see
why this rule works.
Example 1.8 Consider 743 >< 10. Writing 743 in expanded notation, we have
743 = 7(102)+ 4(101) + 3(1), '
Multiplying each term in this equation by 10, we obtain
143 X 10 = mo X 101) +4(1o ><1o1)+ s(1o X 1)
= 1(1o3)+4(1o2)+s(10)
= r(1o3)+ 4(102) + s(1o‘)+ 0(1).
Thus the place value of each digit has been multiplied by 10 and the the entry in the units column
is now O. E1
A very similar thing happens when we multiply a binary number by the base number 2 = (10);.
As we have seen in Example 1.4, the binary number (CZ4Cl3£1gL7.1l1())2 can be written in expanded
notation as
a4(24) + a3(23) + flg(22) + al(21) + 110(1).
Multiplying this number by 2, we obtain
a4(2 >< 24) + a3(2 >< 2*’) + z1;(2 >< 22) + al(2 >< 21) + aD(2 ><1)
= a4(25) + a3(24) + £Zg(23) + al(22) + aq(2)
= (“(25) + ago“) + a2(23) + @102) + (“(21) + 0(1).
Thus the place value of each bit has been multiplied by 2 and so the bits all move one place to the
left and the entry in the units column is 0. Thus we have the following rule:
Rule 1.3 To multiply by (10); in binary, we move each bit one place to the left and enter (I in the
2° column.
Example 1.9 Multiplying (1101); >< (10);, the rule gives (11010); Checking in decimal, we have
calculated 13 x 2 and obtained 26. D
We can extend the rule for multiplying by 2 to multiplying by any power of 2. For example, since
multiplying by 4 is the same as multiplying b“y 2 and then multiplying the result by 2 again, we
move each bit two places to the left and enter zeros in both the 21 and the 2° columns. Thus the
rule for multiplying by 4 : (100); in binary is exactly the same as the rule for multiplying by 100
in decimal. Similarly, the rule for multiplying by 8 = (1000); in binary is the same as the rule for
multiplying by 1000 in decimal, and so on. 3
Example 1.10 \Ve multiply (1011)-_» >< (101,? Note first that (101); : (100); + (1)9. So we



CHAPTER 1. NUMBERS SYSTEMS T
perforni two multiplications and add them together.
>- >-
G;-1»--Q
>-I©©>-4
9-IQ?--*>—l
101
(1011); x (100);
(1011); >< (1);
1 bits carried forward in the sum
1 1 0 1 1 1 (1011); >< (101);
Thus we have (1011); >< (101); = (l.10l11);3. We leave you to check this is correct by finding the
decimal equivalent. E!
Binary division
We first note that dividing by 2 is just the reverse of the process of multiplying by 2 and so we
have the following rule. _ '
Rule 1.4 To divide by (10); in binary, we move each bit one place to the right and the entry in
the units column becomes the remainder.
Example 1.11 (110111); + (10)/1 I (11011)g, remainder (1);. E1
The number we are dividing by is called the divisor and the number of complete times it divides
is called the quotient. In the example above, the divisor is (10); and the quotient is (1101l.)g.
We can easily extend Rule 1.4 to powers of 2, as illustrated in the following example.
Example 1.12 (110111);-I~ (1000); : (1lO)g, remainder (11l)2. El
Note that to retrieve the number we are dividing into, we reverse the process by multiplying the
quotient by the divisor and adding the remainder. Thus the reverse of the process in Example 1.11
is
(110111); = (l101l)g >< (10); + (1);.
The process of long division in the binary system is very similar to long division in the decimal
system, but simplified by the fact that when we divide a binary number by a divisor with the some
number of bits, the quotient is either (1); or (0)2. In the latter case, when we bring down the next
hit, we will always obtain a quotient of (1)-1. The process is illustrated in the following example.
Example 1.13 We calculate (101001); + (11),
Step 1 We try dividing (11); into the first two bits. Since (11); is greater than (10);>, the quotient
is 09 and we bring down the next bit, so that we are now dividing (11); into (101)2. The
quotient this time is (1);, which we record in the quotient line above the third bit. We now
subtract (11); X (1)2 = (11); from the first three bits to obtain the remainder (10);.
»-->-
O»-v->—¢
1110 001
Step 2 We bring down the next bit. We are now dividing a 2-bit number into a 3-bit number, so
the quotient must be (1);. We repeat the process described in Step 1.
>-»—1
1-10>-0-~>~
1-0 P—\
1110 001
?'T

CHAPTER 1. NUMBERS SYSTEMS 8
Step 3 We bring down the next bit. Then since (ll); is greater than (10);, we record (02) in
the quotient row and bring down the last bit. The quotient must be (1)2 this time, so We
record (1); above the last bit in the quotient row and subtract (ll); from (101]g to give the
remainder.
>—~>-\
>-lo»->-->-4
\-0 >-»
O
>—-
1110001
1 O 1
1 1
1 0 '
Thus we have obtained (101001); + (11)-1 : (1101); with remainder (10);. Checking in decimal,
we have divided 41 by 3 to obtain 13 with remainder 2. E3

