\documentclass[12pt]{article}

\usepackage{amsmath}
\usepackage{amssymb}

%opening
\title{Mathematics for Computing}
\author{Hibernia College}

\begin{document}

\maketitle

\section{Numbers and Number Systems}
\begin{itemize}
\item Binary to Decimal conversion
\item Decimal to Binary conversion
\item Decimal to Hexadecimal conversion
\item Hexadecimal to Decimal conversion
\item Floating Point Notation
\item Membership Tables
\end{itemize}
\section{Number Systems}
\begin{enumerate}
\item Binary
\item DEcimal
\item Hexadecimal
\item Octal
\end{enumerate}
\begin{itemize}
\item Decimal Number - What you are probably used to.
\item Binary - Zeroes and Ones.
\item Hexadecimal - examples: RGB and Colours.
\end{itemize}



%-------------------------------------------------------%

\section{Decimal to Binary Conversion(1.4.1)}

\Large{\begin{itemize}
\item Continuously divide the decimal number by 2.
\item Keep record of the remainder, either 0 or 1.
\item The sequence of remainders is the binary number required.
\end{itemize}}
\section{Binary Conversation}
The binary number 100101 is converted to decimal form as follows:

\[
100101_2 = [ ( 1 ) \times 2^5 ] + [ ( 0 ) \times 2^4 ] + [ ( 0 ) \times 2^3 ] + [ ( 1 ) \times 2^2 ] + [ ( 0 ) \times 2^1 ] + [ ( 1 ) \times 2^0 ]  
\]
\[
100101_2 = [ 1 \times 32 ] + [ 0 \times 16 ] + [ 0 \times 8 ] + [ 1 \times 4 ] + [ 0 \times 2 ] + [ 1 \times 1 ] 
\]
\[
100101_2 = 37_{10}
\]

% http://www.binarymath.info/practice-exercises.php

\newpage
\section{Binary Arithmetic}



\subsection{Binary Substraction}
\textbf{Exercises:}
6. 110 - 10 \\
7. 101 - 11 \\
8. 1001 - 11 \\
9. 1101 - 11 \\
10. 10001 - 100 \\
\newpage
% http://www.mathsisfun.com/binary-decimal-hexadecimal.html


\end{document}
